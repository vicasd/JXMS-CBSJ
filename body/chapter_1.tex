% !Mode:: "TeX:UTF-8" 
\section{中国传统计时单}
1.中国传统计时单位。把一昼夜平分为十二段,每段叫做一个时辰,合现在的两小时。十二个时辰分别以地支为名称,从半夜起算,半夜十一点到一点是子时,中午十一点到一点是午时。 宋 苏轼 《申三省起请开湖六条状》:“今来所创置钤辖司前一闸,虽每遇潮上,闭得一两时辰……而公私舟舡欲出入闸者,自须先期出入,必不肯端坐以待闭闸。” 清 孔尚任 《桃花扇·媚座》:“昼短夜长,差了三个时辰了。” 曹禺 《王昭君》第二幕:“半个时辰前,他陪伴天子酣宴。”
2.泛指时刻或时间。 金董解元 《西厢记诸宫调》卷一:“没一个时辰儿不挂念,没一个夜儿不梦见。”《西游记》第四九回:“这等干,只是忒费事,耽搁了时辰了。” 魏巍 《东方》第四部第二三章:“这就叫:不是不报,时辰不到。” [1] 
时间来历编辑 播报
时辰亭
时辰亭
古代劳动人民最初描述时间主要参照显而易见的天象、动物生物钟和日常作息,比如鸡鸣、平旦、朝食、日中、人定等,后来逐渐改用授时设备读数来描述,后来也逐渐给这些授时设备读数配上了五行。比如,给甲乙两字配上木(4:48-9:36)、给丙丁两字配上火(9:36-14:24)、给戊己两字配上给土(14:24-19:12)、给庚辛两字配上金(19:12-24:00)、给壬癸两个字配上水(0:00-4:48)。这些附会并无实际意义。
古时
今时
时间段〔使用同一地支表述的时间范围〕
子时
0:00
唐前:0:00-2:00
唐后:23:00-1:00
丑时
2:00
唐前:2:00-4:00
唐后1:00-3:00
寅时
4:00
唐前:4:00-6:00
唐后:3:00-5:00
卯时
6:00
唐前:6:00-8:00
唐后:5:00-7:00
辰时
8:00
唐前8:00-10:00
唐后7:00-9:00
巳时
10:00
唐前10:00-12:00
唐后9:00-11:00
午时
12:00
唐前12:00-14:00
唐后11:00-13:00
未时
14:00
唐前14:00-16:00
唐后13:00-15:00
申时
16:00
唐前16:00-18:00
唐后15:00-17:00
酉时
18:00
唐前18:00-20:00
唐后17:00-19:00
戌时
20:00
唐前20:00-22:00
唐后19:00-21:00
亥时
22:00
唐前22:00-24:00
唐后21:00-23:00
古代的更编辑 播报
古代计时装置日晷
古代计时装置日晷
古代的更是按时间算的。
19:00-21:00为一更,
21:00-23:00为二更,
23:00-01:00为三更,
01:00-03:00为四更,
03:00-05:00为五更。
注:此处“更”在古汉语中读“jing”。
周易编辑 播报
在论断八字命理时,往往有时不太准确,多半是因时辰有误所致。在我们生活中,全国都是统一使用北京标准时间。我国南北区域时差不算太大,而东西区域相差较大,而论命时就要严格按照出生地时间为太极点论断。
时间差求法:北京时间,准确地说是指在东经120度所在地区的当地时间,由于北京大约在东经117度左右,故此北京时间12:12才是标准的北京当地时间的12点正。而或东或西的不同地区就要根据当地所在区域内的东经度数推算。地球自西向东自转一周(360度),需24小时,每4分钟1度(1度距离相差4分钟,“东加西减”原则)。其他地区,依此推算即可。例如:当北京时间12:00时,黑龙江东半部(东经130度)就是12:40,{12:00+(130-120)×4分=12:40};西藏西半部和新疆中部(东经90度)就是10:00,{12:00-(120-90)×4分=10:00}。东经120度北京时间12点与其他地区所在东经度数时间对照表:
时辰钟
时辰钟
东经│ 75 │ 80 │ 85 │ 90 │ 95 │ 100 │ 105
时间│ 9:00 │ 9:20 │9:40 │10:00 │10:20 │10:40 │11:00
东经│ 110 │ 115 │ 120 │ 125 │ 130 │ 135 │ 140
时间│ 11:20│11:40 │12:00 │12:20 │12:40 │13:00 │13:20
再如古代“占星术”起命宫时,先看太阳在什么宫,再以生时加在太阳所在宫里起顺数至卯位上。由此可以看出,古代星命学家推断一个人的命运时,都是以太阳作为命之所在,即日出于卯而没于酉。由于不同经度的地区见到太阳的时间是不同的,故而以出生时间论命时必须以当地时间为准。
时辰历史编辑 播报
十二时辰制
西周时就已使用。汉代命名为夜半、鸡鸣、平旦、日出、食时、隅中、日中、日昳、晡时、日入、黄昏、人定。又用十二地支来表示,以夜半二十三点至一点为子时,一至三点为丑时,三至五点为寅时,依次递推。
【子时】夜半,又名子夜、中夜:十二时辰的第一个时辰。(23时至01时)。
【丑时】鸡鸣,又名荒鸡:十二时辰的第二个时辰。(01时至03时)。
【寅时】平旦,又称黎明、早晨、日旦等:时是夜与日的交替之际。(03时至05时)。
【卯时】日出,又名日始、破晓、旭日等:指太阳刚刚露脸,冉冉初升的那段时间。(05时至07时)。
【辰时】食时,又名早食等:古人“朝食”之时也就是吃早饭时间,(07时至09时)。
【巳时】隅中,又名日禺等:临近中午的时候称为隅中。(09 时至11时)。
【午时】日中,又名日正、中午等:(11时至13时)。
【未时】日昳,又名日跌、日央等:太阳偏西为日昳。(13时至15时)。
【申时】晡时,又名日铺、夕食等:(15时至17时)。
【酉时】日入,又名日落、日沉、傍晚:意为太阳落山的时候。(17时至19时)。
【戌时】黄昏,又名日夕、日暮、日晚等:此时太阳已落山,天将黑未黑。天地昏黄,万物朦胧,故称黄昏。(19时至21时)。
【亥时】人定,又名定昏等:此时夜色已深,人们已经停止活动,安歇睡眠了。人定也就是人静。(21时至23时)。
二十四时辰制
宋以后把十二时辰中每个时辰平分为初、正两部分,这样,子初、子正、丑初、丑正......依次下去,恰为二十四时辰,同现在一天二十四小时时间一致。
十时辰制
出现于先秦。昼夜各五分。据《隋书.天文志》,昼为朝、禺、中、晡、夕;夜为甲、乙、丙、丁、戊(后用五更来表示)。即更点———古代把晚上戌时作为一更,亥时作为二更,子时作为三,丑时为四,寅时为五更。把一夜分为五更,按更击鼓报时,又把每更分为五点。每更就是一个时辰,相当于现在的两个小时,即120分钟,所以每更里的每点只占24分钟。由此可见“四更造饭,五更开船”相当于现在的“后半夜1时至3时做饭,3时至5时开船”。“五更三点”相当于现在的早晨3时又72分钟,即4时12分,“三更四点”相当于现在的晚上11时又96分钟,即0时36分。
十五时辰制
其名称为晨明、朏明、旦明、蚤(早)食、宴(晚)食、隅中、正中、少还、铺时、大还、高舂、下舂、县(悬)东、黄昏、定昏。参阅《淮南子.天文训》。
\section{夜分成均衡的一百刻。其产生}
即把昼夜分成均衡的一百刻。其产生与漏刻的使用有关。可能起源于商代。汉时曾把它改造为百二十刻,南朝梁改为九十六刻、一百零八刻,几经反复,直至明末欧洲天文学知识传入才又提出九十六刻制的改革。
古代意义
而又常见常用的有关名称亦颇不少。一般地说,日出时可称旦、早、朝、晨,日入时称夕、暮、晚。太阳正中时叫日中、正午、亭午,将近日中时叫隅中,偏西时叫昃、日昳。日入后是黄昏,黄昏后是人定,人定后是夜半(或叫夜分),夜半后是鸡鸣,鸡鸣后是昧旦、平明——这是天已亮的时间。古人一天两餐,上餐在日出后隅中前,这段时间就叫食时或早食;晚餐在日昃后日入前,这段时间叫晡时。
相关内容编辑 播报
在中国古代,人们用“铜壶滴漏”的方法计时,把一昼夜分为十二时辰,即子、丑、寅、卯、辰、巳、午、未、申、酉、戌、亥,对应于今天的二十四小时。半夜十一点到一点的时间为子时,一点到三点为丑时,三点到五点为寅时,其余的以此类推。古代的一个时辰相当于今天的两个小时,所以,当钟表刚刚传入中国时,就有人把一个时辰叫做“大时”,新时间的一个钟点叫做“小时”。以后,随着钟表的普及,“大时”一词也就消失了,而“小时”却沿用至今。
刻———古代用漏壶计时。
漏壶分播水壶和受水壶两部。播水壶分二至四层,均有小孔,可滴水,最后流入受水壶,受水壶里有立箭,箭上刻分100刻,箭随蓄水逐渐上升,露出刻数,以显示时间。而一昼夜24小时为100刻,即相当于现在的1440分钟。可见每刻相当于现在的14.4分钟。所以“午时三刻”相当于现在的中午11时 43.2分
旧小说有"午时三刻开斩"之说,意即,在午时三刻钟(差十五分钟到正午)时开刀问斩,此时阳气最盛,阴气即时消散,此罪大恶极之犯,应该"连鬼都不得做",以示严惩。阴阳家说的阳气最盛,与现代天文学的说法不同,并非是正午最盛,而是在午时三刻。古代行斩刑是分时辰开斩的,亦即是斩刑有轻重。一般斩刑是正午开刀,让其有鬼做;重犯或十恶不赦之犯,必选午时三刻开刀,不让其做鬼。皇城的午门阳气也最盛,不计时间,所以皇帝令推出午门斩首者,也无鬼做。
刻制
古代一昼夜划为十二个时辰,又划为九十六刻。
一刻约15分钟。
时辰与养生编辑 播报
时辰养生歌诀
总结一首歌诀,有助于您更好地遵照12个时辰来养生:
寅时天亮便起身,喝杯开水楼下行;
定时如厕轻如许,卯时晨练最宜人;
辰时看书戏幼孙,巳时入厨当灶君;
午时进餐酒少饮,未时午休要抓紧;
申时读报写诗文,酉时户外看流云;
戌时央视新闻到,闭目聆听好养神;
亥时过半快洗漱,子时梦中入画屏;
丑时小解一时醒,轻摩“三丹”气血盈;
脉络通畅心如水,一觉睡到金鸡鸣。
子时——睡觉保护阳气
半夜11点到凌晨1点的时候叫子时,这个时候是一天当中太极生命钟的阴极,按照阴阳消长的规律,这个时候阴气是最重的,而阴是主睡眠的,那么我们就要遵从这个阴阳的消长规律,在这个时候我们要处于熟睡的状态。大家注意,不是这个时候我们才上床,而是这个时候我们应该已经处在熟睡状态了。那么什么时候该上床呢?应该在10点半左右。子时的时候,是胆经值班。胆气在一阳生的时候,是刚刚长出来的阳气,还很微弱,我们要特别保护这个阳气,应该怎么保护呢?最好用睡觉来保护,所以夜半的时候,我们就不要再去跳啊唱啊的,而应该睡觉。这时候要开始养阳气,而养阳气要从它微小的时候就要保护它。
丑时——肝经造血时间
丑时是凌晨1点到凌晨3点的时候,是肝经值班。肝经是主生发的,这个时候的阳气比胆经值班的时候要生得大一点了。肝脏要解毒、要造血,就是在这个时候进行,所以半夜里,千万别去酗酒,千万别沉迷于游戏了。这个时候人体得休息,肝还要工作。有肝病的人多是爱熬夜的人,因为半夜肝要造血、要解毒,如果不给它喘息的机会,自然就容易发病。
寅时——号脉的最好时机
夜里3点到夜里5点是什么时候呢?这时候叫做平旦。因为此时天气要开始平衡了,阴阳开始平衡了。此时肺经值班。此时,天刚刚亮,这时候中医号脉是最准的时候。我们可以看你的脉硬不硬,脉硬呢,40岁以上的人要考虑是否得了高血压;二三十岁的如果脉紧,可能是工作压力太大,还可能是有焦虑症。又紧又硬的脉叫做弦脉,如果是弦脉就要考虑你是不是有高血脂、动脉硬化了。
卯时——空腹喝水,排出毒素
卯时是大肠值班的时候,它是早上5到7点钟。12时辰养生有个重点,就是卯时起床后要喝一杯空腹水,有便秘的人这样做就可以帮助你减轻便秘。因为大肠在此时精气开始旺盛,大肠一鼓动,再加上你的水的帮助,大便就下来了,就能帮助你解毒。要知道大便里的毒占人体所有毒的50\%。卯时气血流注于大肠经,卯时在天地之象代表天门开,代表二月,万物因阳气的生发冒地而出,故是排便的最佳时机。中医认为“肺与大肠相表里”,寅时肺气实了,卯时才能正常地大便。
辰时——早餐营养要均衡
到辰时的时候,是胃经值班了。这个时候是7到9点,7点钟我们要吃早饭了,而这个时候是胃经值班,所以,胃在此时是最容易接纳食物的。早餐一定要有动物蛋白,要有一味荤,比如说你咸菜就泡饭就不行,得加上一点有动物蛋白的东西,要有一点肉,或者鸡蛋。
巳时——工作学习的第一个黄金时间
接下去上午9到11点是巳时,这时是脾经值班的时候。脾经是主消化的,这个时候,它要吸收营养。而这个时候也是大脑最具活力的时候,是人的一天当中的第一黄金时间,是老人锻炼身体的最好时候,是上班族最出效率的时候,也是上学的人效率最高的时候。所以,我们必须吃好早饭,保证脾经有足够的营养吸收,这样,大脑才有能量应付日常的运转。
午时——睡好午觉养阳气
午时11点到下午1点的时候,是心经值班。这个时候大家要注意,心经值班的时候我们要吃午饭、睡午觉,因为按照太极阴阳气化规律,这个时候阳气最旺。《黄帝内经》说,阴是主内的,是主睡觉;阳是主外的,主苏醒。午时是阳气最盛的时候,我们吃完午饭稍事休息继续工作,这个时候也出效率。阳虚的人这个时候就要好好地睡上一觉,最养阳气。那么阳气不虚也不盛,正常的人怎么办呢?我们午时只需休息半小时到一小时,养养我们的心经。因为我们的心脏很累,除非你是身体很强的人,你可以不睡午觉,一般的话我们还是要睡午觉。睡午觉要平躺,这样可以让大脑和肝脏得到血液,有利于大脑养护。《黄帝内经》有一句话,叫做“卧,则血归于肝”。你要平躺,这样血液才养肝。肝脏对人体有个重要的分布血液的作用,你睡午觉起来以后,肝脏就可以把血液输送到你的大脑,保证你的工作效率,所以午觉最好要平躺。平躺还有一个什么好处呢?就是保护你的颈椎、你的腰椎。人的骨架就像房子的柱梁,一天到晚地撑着身体,何其累也,所以你看年纪老的人,骨头的病就多。午睡躺下,颈椎可以得到休息,腰也可以歇一会儿,长此以往,你就不会得腰椎增生、颈椎病,也不会得坐骨神经痛等病,所以午时午休是很重要的。
未时——保护血管多喝水
午时过了以后,下午1点到下午3点,就到了未时,这时小肠经值班。小肠经把食物里的营养都吸收得差不多了,都送到了血液里边,血液里边就满满当当的,就像上下班时候街上的车,十分拥挤。这个时候我们必须要喝一杯空腹水,或者是茶也行,这是用来稀释你的血液。因为人体这个时候血液营养很高,很粘稠,所以要稀释血液,这样才能起到保护血管的作用。
申时——工作学习的第二个黄金时间
未时过了就到申时了。下午的3点到下午5点之间,大家要注意了,这是我们的第二个黄金时间。这个时候小肠经已经把中午饭的营养都送到大脑了,大脑这时候精力很好,要抓紧工作,提高效率。
酉时——预防肾病的最佳时期
那么到了酉时,也就是傍晚的5点到傍晚7点,这时候是肾经值班,我们要再喝一杯水。这一杯水非常重要,它可以帮我们把毒排掉,还可以清洗你的肾和膀胱,让我们不得肾结石,不得膀胱癌,不得肾炎。
戌时——工作学习的第三个黄金时间
再往下,到了戌时,也就是晚上的7点到9点,此时是心包经值班。心包经值班的时候呢,我们的心气比较顺了。这个时候是我们一天当中的第三个黄金段,这个时间你可以学习,可以去散步去锻炼身体。但是,当心包经值班时间快结束时,可能是你散步回来以后,你需要再喝一杯淡茶水或者是水,让你的血管保持通畅。
亥时——准备休息
然后就到了亥时,亥时就是晚上的9点到11点,这时候应该休息,准备睡觉,或者是夫妻融洽等等,这都是最佳时间。到10点半你就一定要上床了。到此为止,12时辰养生就全部介绍完了。它的规律就是要按照经络和脏腑,还有阴阳气化来进行养生。
时刻一一时晨分为上时刻,中时刻,下时刻:一个时刻是多长时间,上时刻为时辰的前40分钟,中时刻为时辰的中间40分钟,下时刻为时辰的后面40分钟。 [1] 